% Magic comments - Informa ao compilador algumas regras de execução, como tipo de compilação ou codificação do texto
% !TeX root = modelo-trabalho-academico-generico.tex
% !TeX encoding = UTF-8
% !BIB TS-program = XeLaTex
% !backend = biber
% ------------------------------------------------------------------------
% ------------------------------------------------------------------------
% abnTeX2: Modelo de trabalho Academico (tese de doutorado, dissertacao de
% mestrado e trabalhos monograficos em geral) em conformidade com 
% ABNT NBR 14724:2018: Informacao e documentacao - Trabalhos academicos -
% Apresentacao
% ------------------------------------------------------------------------
% ------------------------------------------------------------------------

\documentclass[
	% -- opções da classe memoir --
	12pt,				% tamanho da fonte
	openright,			% capítulos começam em pág ímpar (insere página vazia caso preciso)
	twoside,			% para impressão em verso e anverso. Oposto a oneside
	a4paper,			% tamanho do papel. 
	% -- opções da classe abntex2 --
	%chapter=TITLE,		% títulos de capítulos convertidos em letras maiúsculas
	%section=TITLE,		% títulos de seções convertidos em letras maiúsculas
	%subsection=TITLE,	% títulos de subseções convertidos em letras maiúsculas
	%subsubsection=TITLE,% títulos de subsubseções convertidos em letras maiúsculas
	% -- opções do pacote babel --
	english,			% idioma adicional para hifenização
	brazil				% o último idioma é o principal do documento
	]{abntex2} % classe abntex2 para escrita de trabalhos academicos - serve apenas para o pre-textual

% ---
% Pacotes básicos 
% ---
\usepackage{graphicx}			% Inclusão de gráficos
\usepackage{microtype} 			% para melhorias de justificação
\usepackage{amsmath,amssymb,unicode-math} % escrita matematica
\usepackage{relsize} % readicionado comando \part não funciona. Ver: https://github.com/abntex/abntex2/issues/109

\usepackage{pdfpages} % para incluir páginas pdf diretamente no documento final, como capa e folha de aprovação

% Estilo das aspas
\usepackage[style=brazilian]{csquotes}

%% --
% Pacotes de citações: 
% --
% Atenção: Ustitlizando o pacote biblatex-abnt e o biber para a compilacao das fontes. Compilar com compilador XeLaTeX para o texto e Biber para as referências.
% -- 

\usepackage[backend=biber,
	% -- opções do pacote biblatex-abnt --
	% style = abnt-ibid %% ABNT - citacoes em notas de notas de rodapé com abreviações latinas
	% style=abnt-numeric %% ABNT - Estilo Numérico
	style=abnt, %% ABNT - Sistema autor-data se utilizar comando \cite ou notas explicativas em rodapé com comando \footcite
	% --
	sccite, % citacoes em versalete
	scbib, % Bibliografia em versalete
	ittitles, % títulos em itálico na bibliografia
	citecount, % especifica quantas vezes a referência foi citada
	justify, % alinhamento justificado na bibliografia
	noslsn, % Oculta as abreviações [s.l. sem local e s.n. sem número] na bibliografia
	repeatfields, % evita traço em autor repetido		
	sorting=nty, % ordem alfabética e por ano
]{biblatex}
\addbibresource{referencias-teste.bib}
% ---
% Personalização do estilo biblatex-abnt para as novas normas ABNT 6023:218
%---
% Adequa as urls de acordo com normas 6023:2018:
\DeclareFieldFormat{url}{\bibstring{urlfrom}\addcolon\addspace \url{#1}}%
\DeclareFieldFormat{urldate}{\bibstring{urlseen}\addcolon\addspace #1}%

% Reseta contadores das notas de rodapé em cada capítulo
%\makeatletter
%\@addtoreset{footnote}{chapter}
%\makeatother

% Comandos para exibir uma caixa colorida de fundo cinza para notas historicas
%\newcommand{\caixabranca}{\fbox}
%\newcommand{\caixa}[1]{\caixabranca{\small{\begin{minipage}{\textwidth-2.5cm}{#1}\end{minipage}}}}

% ---
% Pacotes adicionais, usados no anexo do modelo de folha de identificação
% ---
\usepackage{supertabular}
\usepackage{multicol}
\usepackage{multirow}
\usepackage{lipsum}				% para geração de dummy text
\usepackage{booktabs,tabularx,rotating}	% para tabelas
\usepackage{multicol}				% tabelas com colunas mescladas
% ---



% ---
% Informações de dados para CAPA e FOLHA DE ROSTO
% ---
\titulo{Modelo de trabalho acad{\^e}mico (monografia, dissertação, tese) com \abnTeX e BibLaTeX-ABNT}
\autor{Estudante}
\local{Belo Horizonte}
\data{\today}
\orientador{prof. James Clerk Maxwell, PhD.}
\coorientador{prof. Oliver Heaviside}
\instituicao{Universidade Federal de Minas Gerais - Escola de Engenharia}
\tipotrabalho{Tese de Doutorado}
% O preambulo deve conter o tipo do trabalho, o objetivo, 
% o nome da instituição e a área de concentração 
\preambulo{Monografia apresentada ao Curso XXX da Universidade Federal de Minas Gerais como parte dos requisitos para a obten{\c c}{\~a}o do Grau de Doutor em XXXX.}
% ---


% ---
% Configurações de aparência do PDF final

% alterando o aspecto da cor azul
\definecolor{blue}{RGB}{41,5,195}

% informações do PDF
\makeatletter
\hypersetup{
     	%pagebackref=true,
		pdftitle={\@title}, 
		pdfauthor={\@author},
    	pdfsubject={\imprimirpreambulo},
	    pdfcreator={LaTeX with abnTeX2},
		pdfkeywords={abnt}{latex}{abntex}{abntex2}{trabalho acadêmico}, 
		colorlinks=true,       		% false: boxed links; true: colored links
    	linkcolor=blue,          	% color of internal links
    	citecolor=blue,        		% color of links to bibliography
    	filecolor=magenta,      		% color of file links
		urlcolor=blue,
		%bookmarksdepth=4
}
\makeatother
% --- 

% --- 
% Espaçamentos entre linhas e parágrafos 
% --- 

% O tamanho do parágrafo é dado por:
\setlength{\parindent}{1.3cm}

% Controle do espaçamento entre um parágrafo e outro:
\setlength{\parskip}{0.2cm}  % tente também \onelineskip

% ---
% compila o indice
% ---
\makeindex
% ---

% ----
% Início do documento
% ----
\begin{document}

% Retira espaço extra obsoleto entre as frases.
\frenchspacing 

% ----------------------------------------------------------
% ELEMENTOS PRÉ-TEXTUAIS
% ----------------------------------------------------------
% \pretextual

% ---
% Capa
% ---
\imprimircapa
% ---

% ---
% Folha de rosto
% (o * indica que haverá a ficha bibliográfica)
% ---
\imprimirfolhaderosto*
% ---


% ---
% Inserir a ficha bibliografica
% ---

% Isto é um exemplo de Ficha Catalográfica, ou ``Dados internacionais de
% catalogação-na-publicação''. 
% Porém, a biblioteca da sua universidade lhe fornecerá um PDF
% com a ficha catalográfica definitiva após a defesa do trabalho. Quando estiver
% com o documento, salve-o como PDF no diretório do seu projeto e substitua todo
% o conteúdo de implementação deste arquivo pelo comando abaixo que está comentado 
% (nao se esqueça de comentar o antigo ambiente de ficha catalográfica):
%
% \begin{fichacatalografica}
%     \includepdf{fig_ficha_catalografica.pdf}
% \end{fichacatalografica}
%
%
% Ou, você poderá também ler os dados da ficha e adicionar no póprio código
% como as palavras-chave, CDU e dimensões do trabalho: 
\begin{fichacatalografica}
	\vspace*{\fill}					% Posição vertical
	\hrule							% Linha horizontal
	\begin{center}					% Minipage Centralizado
	\begin{minipage}[c]{12.5cm}		% Largura
	
	\imprimirautor
	
	\hspace{0.5cm} \imprimirtitulo  / \imprimirautor. --
	\imprimirlocal, \imprimirdata-
	
	\hspace{0.5cm} \pageref{LastPage} p. : il. (algumas color.) ; 30 cm.\\
	
	\hspace{0.5cm} \imprimirorientadorRotulo~\imprimirorientador\\
	
	\hspace{0.5cm}
	\parbox[t]{\textwidth}{\imprimirtipotrabalho~--~\imprimirinstituicao,
	\imprimirdata.}\\
	
	\hspace{0.5cm}
		1. Palavra-chave1.
		2. Palavra-chave2.
		I. Orientador.
		II. Universidade xxx.
		III. Faculdade de xxx.
		IV. Título\\ 			
	
	\hspace{8.75cm} CDU 02:141:005.7\\
	
	\end{minipage}
	\end{center}
	\hrule
\end{fichacatalografica}
% ---

% ---
% Inserir folha de aprovação
% ---

% Isto é um exemplo de Folha de aprovação, elemento obrigatório da NBR
% 14724/2011 (seção 4.2.1.3). Você pode utilizar este modelo até a aprovação % do trabalho. Após isso, substitua todo o conteúdo deste arquivo por uma % imagem da página assinada pela banca com o comando abaixo:
%
% \includepdf{folhadeaprovacao_final.pdf}
%
\begin{folhadeaprovacao}
% 
   Monografia defendida e aprovada, em $XX$ de $XX$ de \imprimirdata, pela comiss{\~a}o avaliadora constitu{\'i}da pelos professores:
   \vspace*{\fill}
   \assinatura{\textbf{\imprimirorientador} \\ Orientador} 
   \vspace*{\fill}
   \assinatura{\textbf{Profa. Mary B. Hesse, Dra.} \\ Convidada}
   \vspace*{\fill}
   \assinatura{\textbf{Prof. William Thomson, Dr.} \\ Convidado}
   \vspace*{\fill}
   %\assinatura{\textbf{Professor} \\ Convidado 3}
  % \vspace*{\fill}
   %\assinatura{\textbf{Professor} \\ Convidado 4}
  % \vspace*{\fill}
      
   \begin{center}
    \vspace*{0.5cm}
    {\large\imprimirlocal}, {\large\imprimirdata}
    \vspace*{1cm}
  \end{center}
  
\end{folhadeaprovacao}
% ---

% ---
% Dedicatória
% ---
\begin{dedicatoria}
   \vspace*{\fill}
   \flushright
   \noindent
   \textit{ Este trabalho é dedicado às crianças adultas que,\\
   quando pequenas, sonharam em se tornar cientistas.} \vspace*{\fill}
\end{dedicatoria}
% ---

% ---
% Agradecimentos
% ---
\begin{agradecimentos}
\noindent Agradeça aqui.

\end{agradecimentos}
% ---

% ---
% Epígrafe
% ---
\begin{epigrafe}
    \vspace*{\fill}
	\begin{flushright}
		\textit{``Matéria é a parte acidental.'' (Oliver Lodge)}
	\end{flushright}
\end{epigrafe}
% ---

% ---
% RESUMOS
% ---

% resumo em português
\setlength{\absparsep}{18pt} % ajusta o espaçamento dos parágrafos do resumo
\begin{resumo}
 \noindent Seu resumo aqui.

 \textbf{Palavras-chaves}: latex. abntex. editoração de texto.
\end{resumo}

% resumo em inglês
\begin{resumo}[Abstract]
 \begin{otherlanguage*}{english}

\noindent This is the english abstract.

   \vspace{\onelineskip}
 
   \noindent 
   \textbf{Key-words}: latex. abntex. text editoration.
 \end{otherlanguage*}
\end{resumo}

% resumo em francês 
%\begin{resumo}[Résumé]
% \begin{otherlanguage*}{french}
%    Il s'agit d'un résumé en français.
% 
%   \textbf{Mots-clés}: latex. abntex. publication de textes.
% \end{otherlanguage*}
%\end{resumo}
%
%% resumo em espanhol
%\begin{resumo}[Resumen]
% \begin{otherlanguage*}{spanish}
%   Este es el resumen en español.
%  
%   \textbf{Palabras clave}: latex. abntex. publicación de textos.
% \end{otherlanguage*}
%\end{resumo}
%% ---
% ---
% inserir lista de ilustrações
% ---
\pdfbookmark[0]{\listfigurename}{lof}
\listoffigures*
\cleardoublepage
% ---

% ---
% inserir lista de tabelas
% ---
\pdfbookmark[0]{\listtablename}{lot}
\listoftables*
\cleardoublepage
% ---

% ---
% inserir lista de abreviaturas e siglas
% ---
\begin{siglas}
  \item[ABNT] Associação Brasileira de Normas Técnicas
  \item[abnTeX] ABsurdas Normas para TeX
\end{siglas}
% ---

% ---
% inserir lista de símbolos
% ---
\begin{simbolos}
  \item[$ \Gamma $] Letra grega Gama
  \item[$ \Lambda $] Lambda
  \item[$ \zeta $] Letra grega minúscula zeta
  \item[$ \in $] Pertence
\end{simbolos}
% ---

% ---
% inserir o sumario
% ---
\pdfbookmark[0]{\contentsname}{toc}
\tableofcontents*
\cleardoublepage
% ---




% ----------------------------------------------------------
% ELEMENTOS TEXTUAIS
% ----------------------------------------------------------
\textual
% ----------------------------------------------------------
% PARTE
% ----------------------------------------------------------
\part{Preparação da pesquisa}
% ----------------------------------------------------------
%
% ---
% Modelo de capitulo com a introducao, objetivos e estrutura do texto
% ---
% ----------------------------------------------------------
% Introdução (exemplo de capítulo sem numeração, mas presente no Sumário)
% ----------------------------------------------------------
\chapter[Introdução]{Introdução}
%\addcontentsline{toc}{chapter}{Introdução}
% ----------------------------------------------------------

\section{Justificativas e Relev{\^a}ncia}
%
Este documento e seu código-fonte são exemplos de referência de uso da classe
\textsf{abntex2} e do pacote \textsf{abntex2cite}. O documento 
exemplifica a elaboração de trabalho acadêmico (tese, dissertação e outros do
gênero) produzido conforme a ABNT NBR 14724:2011 \emph{Informação e documentação
- Trabalhos acadêmicos - Apresentação}.
A expressão ``Modelo Canônico'' é utilizada para indicar que \abnTeX\ não é
modelo específico de nenhuma universidade ou instituição, mas que implementa tão
somente os requisitos das normas da ABNT. Uma lista completa das normas
observadas pelo abntex.\footcite{boyle1772}

Sinta-se convidado a participar do projeto \abnTeX!\footcite{herao}
%
\section{Metodologia}

\lipsum[1]

\section{Objetivos}

\lipsum[7]

\lipsum[8]

\section{Organiza{\c c}{\~a}o e estrutura}

\lipsum*[9-11]

\begin{itemize}
\item item;
\item item;
\item item;
\item item;
\end{itemize}

\section{Cronograma}

Esta seção deve constar somente no projeto de monografia. Não deve aparecer na versão final do texto.

% Please add the following required packages to your document preamble:
% \usepackage[table,xcdraw]{xcolor}
% If you use beamer only pass "xcolor=table" option, i.e. \documentclass[xcolor=table]{beamer}

% Please add the following required packages to your document preamble:
% \usepackage[table,xcdraw]{xcolor}
% If you use beamer only pass "xcolor=table" option, i.e. \documentclass[xcolor=table]{beamer}

Um exemplo de cronograma das atividades é proposto na tabela \ref{tab:cronograma}.\footnote{Voc{\^e} pode elaborar também tabelas online, gerando o código em \LaTeX. Após isso, basta copiar e colar o código aqui. Um exemplo de site é o ``Table Generator''\url{http://www.tablesgenerator.com/}.}
  
\begin{table}[h]
\ABNTEXfontereduzida
\caption[Cronograma das atividades]{Cronograma das atividades de elaboração da monografia.}
\label{tab:cronograma}
\begin{minipage}{0.3\textwidth}
    \centering
\begin{tabular}{|l|l|l|l|l|l|l|l|l|l|l|l|l|l|l|l|l|}
\hline
                             & \multicolumn{16}{c|}{Meses}                                                   \\ \hline
Atividades (Etapas)          & 01 & 02 & 03 & 04 & 05 & 06 & 07 & 08 & 09 & 10 & 11 & 12 & 13 & 14 & 15 & 16 \\ \hline
1. Estudo da teoria          & X  & X  & X  & X  & X  &    &    &    &    &    &    &    &    &    &    &    \\ \hline
2. Atualização bibliográfica & X  & X  & X  & X  & X  & X  & X  &    &    &    &    &    &    &    &    &    \\ \hline
3. Seleção de Material       & X  & X  & X  & X  & X  & X  & X  & X  & X  &    &    &    &    &    &    &    \\ \hline
4. Elaboração da monografia  &    &    &    &    & X  & X  & X  & X  & X  & X  & X  & X  & X  & X  & X  &    \\ \hline
5. Elaboração de Artigo      &    &    &    &    &    &    &    &    & X  & X  & X  & X  & X  & X  & X  &    \\ \hline
6. Defesa da monografia      &    &    &    &    &    &    &    &    &    &    &    &    &    &    &    & X  \\ \hline
\end{tabular}
  \end{minipage}
\end{table}
% ---
% Capitulo com exemplos de comandos inseridos de arquivo externo 
% ---
%\include{capitulos/capitulo-abntex2-modelo-include-comandos}
% ---
% ----------------------------------------------------------
% PARTE
% ----------------------------------------------------------
%\part{Referenciais teóricos}
% ----------------------------------------------------------
% ---
% Capitulo de revisão de literatura
% ---
%\include{capitulos/capitulo-revisao-literatura}
% ----------------------------------------------------------
% PARTE
% ----------------------------------------------------------
%\part{Resultados}
% ----------------------------------------------------------
% ---
% primeiro capitulo de Resultados
%\include{capitulos/capitulo-resultados}
% ----------------------------------------------------------
% Finaliza a parte no bookmark do PDF
% para que se inicie o bookmark na raiz
% e adiciona espaço de parte no Sumário
% ----------------------------------------------------------
%\phantompart
% ---
% Insere arquivo de Considerações Finais ou Conclusões
% ---
%\include{capitulos/capitulo-consideracoes-finais}
% ----------------------------------------------------------
% ELEMENTOS PÓS-TEXTUAIS
% ----------------------------------------------------------
\postextual
% ----------------------------------------------------------

%% ----------------------------------------------------------
%% Referências bibliográficas
%% ----------------------------------------------------------
% toca nome de bibliografia para ``Referências''
\printbibliography[title=Refer{\^e}ncias]

% ----------------------------------------------------------
% Glossário
% ----------------------------------------------------------
%
% Consulte o manual da classe abntex2 para orientações sobre o glossário.
%
%\glossary

% ----------------------------------------------------------
% Apêndices
% ----------------------------------------------------------
%(Lembre-se: Apendices são de autoria do próprio autor do texto. 
% Anexos são elementos de autorias de outros, que o autor do texto julga interessante apresentar)
% ---
% Inicia os apêndices: 
% ---
\begin{apendicesenv}

% Imprime uma página indicando o início dos apêndices
\partapendices
% ---
% Insere arquivo com os apendices A e B
\include{capitulos/capitulo-apendice1e2}
\end{apendicesenv}
% ---

% ----------------------------------------------------------
% Anexos
% ----------------------------------------------------------
%(Lembre-se: Apendices são de autoria do próprio autor do texto. 
% Anexos são elementos de autorias de outros, que o autor do texto julga interessante apresentar)
% ---
% Inicia os anexos
% ---
\begin{anexosenv}

% Imprime uma página indicando o início dos anexos
\partanexos

% ---
% Insere arquivo com os anexos 1, 2 e 3
\include{capitulos/capitulo-anexos-1-2-3}
% ---
\end{anexosenv}

%---------------------------------------------------------------------
% INDICE REMISSIVO
%---------------------------------------------------------------------
%\phantompart
\printindex
%---------------------------------------------------------------------

\end{document}
