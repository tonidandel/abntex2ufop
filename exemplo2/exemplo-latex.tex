\documentclass[a4paper,12pt]{article}% Seu arquivo fonte precisa conter
\usepackage[brazil]{babel}           % estas quatro linhas
\usepackage[utf8]{inputenc}          % alem do comando \end{document}
\usepackage{graphicx}			% Inclusão de gráficos
\usepackage{amsfonts}

\begin{document}

\section{Uma visão inicial do LaTeX}  \label{sec:01} 
Na equação \ref{eq:01} , pode-se observar que trata-se de uma reta. Na figura \ref{fig:01}, tem-se a imagem da gloriosa Escola de Minas.
\begin{equation}
f(\alpha) = \frac{1}{\sqrt{\pi}} \int \limits_{0}^{\infty} e^{-\alpha^{2}} d \alpha \,,
\label{eq:01}
\end{equation}





\begin{figure}
\centering	
\includegraphics[scale=0.2]{logo-unidade.pdf}
\caption{Viva a gloriosa Escola de Minas}
\label{fig:01}
\end{figure}


segundo a tabela \ref{tabela:01}, tem-se que:


\begin{table}[]
	\centering
	\caption{Esta é uma tabela legal}
	\label{tabela:01}
	\begin{tabular}{|l|c|c|c|c|}
		\hline
		\textbf{Fontes de variação} & \textbf{$\sum$ de quadrados}                                                                                        & \textbf{Graus de Lib} & \textbf{média quadrática}    & \textbf{$F_{0}$}                             \\ \hline
		\textbf{regressão}          & $(S_{xy}/S_{xx})$                                                                                                   & $1$                   &                              &                                              \\ \hline
		\textbf{Tratamentos}        & \begin{tabular}[c]{@{}c@{}}$SS'_{E}- SS_{E} - $\\ $(S_{xy})^{2}/S_{xx} - [E_{yy}-(E_{xy})^{2}/E_{xx}]$\end{tabular} & $a-1$                 & $\frac{SS'_{E}-SS_{E}}{a-1}$ & $\frac{1}{MS_{E}}\frac{SS'_{E}-SS_{E}}{a-1}$ \\ \hline
		\textbf{Erro}               & $SS_{E} = E_{yy}-(E_{xy})^{2}/E_{xx}$                                                                               & $a(n-1)-1$            & $\frac{SS_{E}}{a(n-1)-1}$    &                                              \\ \hline
		\textbf{Total}              & S\_\{yy\}                                                                                                           & $an-1$                &                              &                                              \\ \hline
	\end{tabular}
\end{table}

\section{Outra visão do LaTeX} \label{sec:02}
alô ainda mundo!
\subsection{Uma subseção}
\subsubsection{isto é demais}
texto.

Segundo informações disponíveis em \cite{fulano2008}, lá não se encontra uma equação do primeiro grau.

\bibliographystyle{abntex2}
\bibliography{referencias}

\end{document}

