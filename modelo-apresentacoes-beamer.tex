% !TeX encoding = UTF-8
% !BIB TS-program = XeLaTex
%% abtex2-modelo-slides.tex, v-1.0 gfabinhomat
%% Copyright 2012-<COPYRIGHT_YEAR> by abnTeX2 group at http://www.abntex.net.br/ 
%%
%% This work may be distributed and/or modified under the
%% conditions of the LaTeX Project Public License, either version 1.3
%% of this license or (at your option) any later version.
%% The latest version of this license is in
%%   http://www.latex-project.org/lppl.txt
%% and version 1.3 or later is part of all distributions of LaTeX
%% version 2005/12/01 or later.
%%
%% This work has the LPPL maintenance status `maintained'.
%% 
%% The Current Maintainer of this work is Fábio Rodrigues Silva, 
%% member of abnTeX2 team, led by Lauro César Araujo. 
%% Further information are available on 
%% http://www.abntex.net.br/
%%
%% This work consists of the files abntex2-modelo-slides.tex, 
%% abntex2-modelo-references.bib and abntex2-modelo-marca.pdf
%%
%% Modelo desenvolvido por Fábio Rodrigues Silva (gfabinhomat@gmail.com)
%% Mais informações podem ser obtidas no guia do usuário Beamer 
%% (http://linorg.usp.br/CTAN/macros/latex/contrib/beamer/doc/beameruserguide.pdf)
%% Informações rápidas podem ser acessadas em http://en.wikibooks.org/wiki/LaTeX/Presentations
%% Alterações - por Danny Tonidandel (tonidandel@gmail.com)
%% 1) Cores do sistema, sobretudo na página de título, com logos das instituições;
%% 2) pacotes para compilação a partir do XeLaTeX e bibliografias reduzidas (ibid., op.cit. etc.) em notas de rodapé, estilo biblatex-abnt;
%% 3) 


% Apresentações em widescreen. Outros valores possíveis: 1610, 149, 54, 43 e 32.
% Por padrão, as apresentações são no formato 4:3 (sem o aspectratio).
\documentclass[aspectratio=169]{beamer}	 	

\usetheme{Pittsburgh}
\usecolortheme{default}
\usefonttheme[onlymath]{serif}			% para fontes matemáticas
% Enconte mais temas e cores em http://www.hartwork.org/beamer-theme-matrix/ 
% Veja também http://deic.uab.es/~iblanes/beamer_gallery/index.html

% Customizações de Cores: fg significa cor do texto e bg é cor do fundo
\definecolor{coolblack}{rgb}{0.0, 0.18, 0.39} % define cor coolblack
\definecolor{cornellred}{rgb}{0.7, 0.11, 0.11} % define cor cornellred
\definecolor{darkelectricblue}{rgb}{0.33, 0.41, 0.47} % define cor darkelectricblue

\setbeamercolor{normal text}{fg=black}
\setbeamercolor{alerted text}{fg=red}
\setbeamercolor{author}{fg=cornellred}
\setbeamercolor{institute}{fg=coolblack}
\setbeamercolor{date}{fg=darkelectricblue}
\setbeamercolor{frametitle}{fg=coolblack}
\setbeamercolor{framesubtitle}{fg=cornellred}
\setbeamercolor{block title}{bg=darkelectricblue, fg=white}		%Cor do título
\setbeamercolor{block body}{bg=lightgray, fg=darkgray}	%Cor do texto (bg= fundo; fg=texto)

% ---
% PACOTES
% ---
%\usepackage[alf]{abntex2cite}		% Citações padrão ABNT
\usepackage[backend=biber,style=abnt-ibid,sccite,citecount,ittitles,scbib,justify,noslsn,repeatfields]{biblatex} % Citações padrão ABNT com estilo biblatex-abnt
\addbibresource{referencias-tese.bib}
\usepackage{relsize} % precisa ser readicionado por que senão o comando \part não funciona no biblatex.  
\usepackage[brazil]{babel}		% Idioma do documento em português
\usepackage{color}			% Controle das cores
\usepackage{amsmath,amssymb,unicode-math} % segundo informacoes, o pacote unicode-math é mais adequado para o compilador xelatex, ao invés do amsfonts
%\usepackage[T1]{fontenc}		% Selecao de codigos de fonte.
\usepackage{graphicx}			% Inclusão de gráficos
%\usepackage[utf8]{inputenc}		% Codificacao do documento (conversão automática dos acentos)
%\usepackage{txfonts}			% Fontes virtuais
\usepackage{csquotes}
% ---

% --- Informações do documento ---
\title{O marco zero da Engenharia Elétrica: modelo, analogia e hipótese além da telegrafia submarina}
\author{Danny Augusto Vieira Tonidandel}
\institute{Universidade Federal de Minas Gerais
	    \par
	    Programa de Pós-Graduação em Engenharia Elétrica
    	\par
		Qualificação (doutorado)}
\date{\small{\today}}
% ---

% ----------------- INÍCIO DO DOCUMENTO --------------------------------------
\begin{document}

% ----------------- NOVO SLIDE -- CAPA --------------------------------
\begin{frame}

\begin{minipage}{1\linewidth}
  \centering
  \begin{tabular}{ccc}
    \begin{tabular}{c}
      \includegraphics[width=1.5cm]{figuras/logo-ufmg.pdf}
    \end{tabular}
    &
    \begin{tabular}{c}
      \textbf{Universidade Federal de Minas Gerais} \\ \textbf{Programa de Pós-Graduação em Engenharia Elétrica}
    \end{tabular}
&
  \begin{tabular}{c}
	\includegraphics[width=1.5cm]{figuras/logo-ppgee.pdf}
\end{tabular}
  \end{tabular}
\end{minipage}

\titlepage % imprime dados como título e autor

\end{frame}

% ----------------- NOVO SLIDE --------------------------------
\begin{frame}{Sumário}
\tableofcontents
\end{frame}

% ----------------- NOVO SLIDE --------------------------------
\section{Introdução}

\begin{frame}{Introdução}


\end{frame}

% ----------------- NOVO SLIDE --------------------------------
\section{Fontes a serem consultadas}
\begin{frame}
\frametitle{Público-Alvo}
\framesubtitle{Usuários já iniciados ao Beamer}

\begin{block}{Título}
 Este modelo foi preparado como uma aplicação do uso do pacote abnTeX2 com o Beamer.\cite{hertz1893}
\end{block}

\begin{itemize}
 \item Alguns comandos são explicados no modelo TEX.
\end{itemize}
\end{frame}

% ----------------- NOVO SLIDE --------------------------------
\begin{frame}{CTAN}

Visite com frequência a página \url{http://www.ctan.org/}. 
Use-a como um guia de orientações gerais.
\vspace{0.7cm}

Outras fontes a serem consideradas:
\begin{enumerate}
 \item \url{http://www.latex-project.org/}
 \item \url{http://www.tex-br.org/}
 \item \url{http://latexbr.blogspot.com.br/}
 \item \url{http://tex.stackexchange.com/}
 \item \url{http://www.tug.org/}
\end{enumerate}

\end{frame}

% ----------------- NOVO SLIDE --------------------------------
\begin{frame}

\begin{figure}[hbtp]
	\centering
	\includegraphics[scale=0.5]{figuras/leydenjar2.pdf}
	\caption{Uma garrafa para o fluido elétrico: esquema de uma garrafa de Leyden com os condutores interno e externo explícitos.} 
	\label{fig:lydenjar2} 
\end{figure}

\end{frame}


% ----------------- NOVO SLIDE --------------------------------
\section{Referências}

% --- O comando \allowframebreaks ---
% Se o conteúdo não se encaixa em um quadro, a opção allowframebreaks instrui 
% beamer para quebrá-lo automaticamente entre dois ou mais quadros,
% mantendo o frametitle do primeiro quadro (dado como argumento) e acrescentando 
% um número romano ou algo parecido na continuação.

\begin{frame}[allowframebreaks]{Referências}
\printbibliography
%\bibliography{referencias-tese}
\end{frame}

% ----------------- FIM DO DOCUMENTO -----------------------------------------
\end{document}